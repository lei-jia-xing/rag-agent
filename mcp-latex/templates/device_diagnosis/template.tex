\documentclass[12pt,a4paper]{article}
\usepackage[margin=2.5cm,top=3cm,bottom=2.5cm]{geometry}
\usepackage{ctex}
\usepackage{graphicx}
\usepackage{booktabs}
\usepackage{longtable}
\usepackage{array}
\usepackage{colortbl}
\usepackage{xcolor}
\usepackage{multirow}
\usepackage{float}
\usepackage{fancyhdr}
\usepackage{titlesec}
\usepackage{setspace}
\usepackage{lastpage}
\usepackage{enumitem}
\usepackage{hyperref}
\usepackage{amsmath}
\usepackage{amssymb}
\usepackage{tabularx}
\usepackage{tcolorbox}
\usepackage{tikz}

% 页面设置
\pagestyle{fancy}
\fancyhf{}
\fancyhead[L]{\small [[report_id]]}
\fancyhead[R]{\small [[diagnosis_date]]}
\fancyfoot[C]{\small 第\thepage\ 页 / 共\pageref{LastPage} 页}

% 标题格式
\titleformat{\section}{\Large\bfseries\color{blue!70!black}}{\thesection}{1em}{}
\titleformat{\subsection}{\large\bfseries\color{teal}}{\thesubsection}{1em}{}
\titlespacing*{\section}{0pt}{12pt}{8pt}
\titlespacing*{\subsection}{0pt}{8pt}{6pt}

% 颜色定义
\definecolor{statusgreen}{RGB}{76,175,80}
\definecolor{statusyellow}{RGB}{255,193,7}
\definecolor{statusred}{RGB}{244,67,54}
\definecolor{headergray}{RGB}{66,66,66}
\definecolor{lightblue}{RGB}{235,245,255}

\begin{document}

% ============= 封面 =============
\begin{titlepage}
  \centering
  \vspace*{2cm}

  {\Huge\bfseries\color{blue!70!black} 设备健康诊断报告}\par
  \vspace{0.5cm}
  {\Large [[title]]}\par

  \vspace{2cm}

  \begin{tabularx}{0.9\textwidth}{@{}lXl@{}}
    \toprule
    \textbf{报告编号} & [[report_id]] & \\
    \midrule
    \textbf{设备名称} & [[device_name]] & \\
    \textbf{设备型号} & [[device_model]] & \\
    \textbf{安装位置} & [[location]] & \\
    \textbf{诊断日期} & [[diagnosis_date]] & \\
    \textbf{数据范围} & [[data_range]] & \\
    \bottomrule
  \end{tabularx}

  \vspace{2cm}

  % 健康评分展示
  \begin{tikzpicture}
    \draw[fill=gray!10, draw=gray!50] (0,0) circle (2.5cm);
    \node at (0,0.3) {\huge\bfseries [[health_score]]};
    \node at (0,-0.5) {\large \textbf{[[health_status]]}};
    \node at (0,-1.2) {\small 风险等级:\textbf{[[risk_level]]}};
  \end{tikzpicture}

  \vspace{1.5cm}
  \small{本报告基于设备监测数据和现场诊断结果生成}

  \vfill
\end{titlepage}

% ============= 1. 执行摘要 =============
\section{执行摘要}
\subsection{诊断概要}
[[abstract]]

% ============= 2. 设备基本信息 =============
\section{设备基本信息}

\begin{table}[H]
\centering
\caption{设备基本信息}
\begin{tabularx}{0.95\textwidth}{@{}lX@{}}
\toprule
\textbf{项目} & \textbf{详情} \\
\midrule
设备名称 & [[device_name]] \\
设备型号 & [[device_model]] \\
安装位置 & [[location]] \\
诊断日期 & [[diagnosis_date]] \\
数据采集范围 & [[data_range]] \\
\bottomrule
\end{tabularx}
\end{table}

\subsection{设备技术参数}
[[technical_parameters]]

\subsection{运行环境}
[[operating_environment]]

\subsection{历史维护记录}
[[maintenance_history]]

% ============= 3. 健康状态评估 =============
\section{健康状态评估}

\begin{table}[H]
\centering
\caption{健康状态总览}
\begin{tabularx}{0.95\textwidth}{@{}llcc@{}}
\toprule
\textbf{评估项目} & \textbf{说明} & \textbf{结果} & \textbf{状态} \\
\midrule
整体健康评分 & 0-100分,分数越高越健康 & \textbf{[[health_score]]分} &
\textbf{[[health_status]]} \\
风险等级 & 低/中/高三个等级 & \textbf{[[risk_level]]} &
\textbf{[[health_status]]} \\
发现问题数 & 当前检测到的主要问题数量 & \textbf{[[issue_count]]}个 & \\
\bottomrule
\end{tabularx}
\end{table}

% ============= 4. 监测数据分析 =============
\section{监测数据分析}

\subsection{监测数据汇总}
[[monitoring_data_summary]]

\subsection{关键指标分析}
[[key_metrics_analysis]]

\subsection{趋势分析}
[[trend_analysis]]

\subsection{异常检测}
[[anomaly_detection]]

% ============= 5. 故障诊断详情 =============
\section{故障诊断详情}

\subsection{故障现象描述}
[[fault_description]]

\subsection{故障原因分析}
[[fault_cause_analysis]]

\subsection{故障定位}
[[fault_location]]

% ============= 6. 风险评估 =============
\section{风险评估}

\subsection{当前风险}
[[current_risks]]

\subsection{潜在风险}
[[potential_risks]]

\subsection{风险控制建议}
[[risk_control]]

% ============= 7. 维护建议 =============
\section{维护建议}

\subsection{紧急处理措施}
\begin{tcolorbox}[colback=red!5!white,colframe=red!75!black,title=\textbf{紧急措施}]
[[urgent_measures]]
\end{tcolorbox}

\subsection{维护计划}
[[maintenance_plan]]

\subsection{备件建议}
[[spare_parts_suggestion]]

% ============= 8. 诊断方法与标准 =============
\section{诊断方法与标准}

\subsection{诊断方法说明}
[[diagnosis_method]]

\subsection{相关标准}
[[related_standards]]

% ============= 9. 结论与建议 =============
\section{结论与建议}
[[conclusion_and_recommendations]]

\end{document}
